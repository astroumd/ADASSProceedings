% super simple template for automated 2020 ADASS manuscript generation from the registration entry
% place this file in your ADASS2020_author_template directory together with your makedefs file
%
% Only few comments here, see the ADASS_template.tex for a more fully commented version, and
% ManuscriptInstructions.pdf if you need more background, and if you even need more, APS's own
% manual2010.pdf has it all!

% Version 8-oct-2020 (Peter Teuben)

${COMMENT}\documentclass[11pt,twoside]{article}
${COMMENT}\usepackage{asp2014}

\aspSuppressVolSlug
\resetcounters

\bibliographystyle{asp2014}

\markboth{${LNAME} and Author2}{${TITLE}}
% remove/add as you need

${COMMENT}\begin{document}

\title{${TITLE}}

% full name: ${FULLNAME}
\author{${FNAME}~${LNAME}${F1} and Sample~Author2${F2}}
\affil{${F1}${INAME}, Institution City, State/Province, Country; \email{${EMAIL}}}
\affil{${F2}Institution Name, Institution City, State/Province, Country}
% remove/add as you need

% remove/add authors as you need
\paperauthor{${FNAME}~${LNAME}}{${EMAIL}}{ORCID}{${INAME}}{Author1 Department}{City}{State/Province}{Postal Code}{Country}
\paperauthor{Sample~Author2}{Author2Email@email.edu}{ORCID_Or_Blank}{Author2 Institution}{Author2 Department}{City}{State/Province}{Postal Code}{Country}
% remove/add as you need

% leave these next few aindex lines commented for the editors to enable them. Use Aindex.py to generate them for yourself.
% first presenting author should be the first entry for bold-facing the author index page-reference
${NOCOMMENT}\aindex{${LNAME},~${FNAMEI}.}
${NOCOMMENT}\aindex{Author2,~S.}
% remove/add as you need

% leave the ssindex lines commented for the editors to enable them, use Index.py to suggest yours
${NOCOMMENT}\ssindex{FOOBAR!conference!ADASS 2020}
${NOCOMMENT}\ssindex{FOOBAR!organisations!ASP}

% leave the ooindex lines commented for the editors to enable them, use ascl.py to suggest yours
${NOCOMMENT}\ooindex{FOOBAR, ascl:1101.010}
  
\begin{abstract}

${ABSTRACT}
  
\end{abstract}

\section{Introduction}

Your abstract currently has ${LENA} characters. For more than 1000
it's possibly too long. Just sayin' Since this paper was written by
some python code, ignore that warning, since you will edit most of
this rubbish away for your final version.


\section{This Template}

To use this 2020 personalized template instead of the {\tt
  ADASS\_template}, copy this file (named something like O3-1.tex,
P5-2.tex, B4.tex, F3.tex, T1.tex, or I2-1.tex) in your local {\tt
  ADASS2020\_author\_template} directory (where you also find the
asp2014.sty, the Makefile etc).  Edit the macros in the {\tt makedefs}
file, and run ``{\tt make}'' and hope for the best.  If that runs into
trouble, check if your version of latex uses a different calling
sequence.  Some instructions are in the Makefile. When you send your
tar file, it's useful to send the {\tt makedefs} file along.

\section{Any Figures?}

This template has no figures. Look for the larger template and
Makefile how to do this. But most importantly, your figures need to
be EPS files, and their names should be ${PCODE}\_f1.eps ,
${PCODE}\_f2.eps  etc.

\section{Any Tables?}

This template has no tables. Look for the larger template
how to do this. 

\section{Final Check}

Please use ``{\tt make check}'' (which runs {\tt PaperCheck.py}) to
check if you can make life for the ADASS editors a little
easier. Pretty Please.

After this ``{\tt make tar}'' will create the correct archive to be sent to
the editors. zip files also work.

\section{Any Photographs?}

If your paper has enough room at the end, the editors may decide to use this
for a conference photograph.

\section{Summary}

This template has no bibtex file.  Look for the larger template and
Makefile how to do this. By default the {\tt Makefile} will create an
empty ${PCODE}.bib. When you add references to this, uncomment the
line \verb+\bibliography+ below, use ``{\tt make pdf}'' to create
your beautifully looking PDF. Only use the
\verb"\citet" and \verb"\citep" macros!

% For example in \citet{PID_adassxxx} it was shown that ...


% \bibliography{${PCODE}}


% if we have space left, we might add a conference photograph here. Leave commented for now.
% \bookpartphoto[width=1.0\textwidth]{foobar.eps}{FooBar Photo (Photo: Any Photographer)}


${COMMENT}\end{document}

